{\color{red} \bf Warning:} This document is written to explain the main
functions of \pkg{pbdSLAP}~\citep{Chen2012pbdSLAPpackage}, version 0.1-0.
Every effort will be made to ensure future versions are consistent with
these instructions, but features in later versions may not be explained
in this document.

Information about the functionality of this package,
and any changes in future versions can be found on website:
\url{http://r-pbd.org/}.



\section[Quick Start]{Quick Start}
\label{sec:quick_start}
\addcontentsline{toc}{section}{\thesection. Quick Start}

The \pkg{pbdSLAP} package serves as a mechanism to utilize a subset of
functions from the ScaLAPACK~\citep{ScaLAPACK1997} library from within
\proglang{R}~\citep{Rcore}, and in particular from the higher level
\proglang{R} packages \pkg{pbdBASE}~\citep{Schmidt2012pbdBASEpackage} and
\pkg{pbdDMAT}~\citep{Schmidt2012pbdDMATpackage}. It allows one to merely
``plug in'' the necessary libraries without needing to do a complicated
system installation.  It is a bundling of the official ScaLAPACK distribution
from the ScaLAPACK Team at netlib (\url{http://www.netlib.org/scalapack/}).
However, it is possible to use other ScaLAPACK libraries instead; see
Section~\ref{sec:external} for details.

The \pkg{pbdSLAP} package consists of a minimum set of double precision
functions from the \pkg{ScaLAPACK} library
for \proglang{R}'s distributed matrix computation. \pkg{ScaLAPACK}
includes many important functions for distributed linear algebra, including
LU factorization, singular value decomposition, etc.
We also include the necessary components of the libraries that our subset of
\pkg{ScaLAPACK} relies on, namely \pkg{BLACS},
\pkg{PBLAS}, \pkg{BLAS}, and \pkg{LAPACK}.  

The system requirements and installation instructions for \pkg{pbdSLAP} are
provided in the following section.
A technical issue for grid information of \pkg{BLACS} is described in
the Section~\ref{sec:blacs}.


\subsection[System Requirements]{System Requirements}
\label{sec:system_requirements}
\addcontentsline{toc}{subsection}{\thesubsection. System Requirements}

\pkg{pbdSLAP} requires \pkg{pbdMPI}~\citep{Chen2012pbdMPIpackage}, which
itself requires a system installation of MPI.
(\url{http://en.wikipedia.org/wiki/Message_Passing_Interface}).
\pkg{pbdSLAP} should also
work with LAM/MPI (\url{http://www.lam-mpi.org/}) and
MPICH2 (\url{http://www.mcs.anl.gov/research/projects/mpich2/}).

\pkg{pbdSLAP} is mainly developed and tested under
{\color{blue} OpenMPI} (\url{http://www.open-mpi.org/}) in
Xubuntu 11.04 and 12.04 systems (\url{http://xubuntu.org/}).
\pkg{pbdSLAP} should also run on other operating systems, such as
Mac with OpenMPI, or Windows with MPICH2 if MPI is installed and launched
properly.  However, we have not extensively tested installation and use of
the \pkg{pbd} toolchain on other platforms. 
The reader is encouraged to report his/her experience with \pkg{pbdSLAP}
on other platforms.


\subsection[Installation and Quick Start]{Installation and Quick Start}
\label{sec:installation}
\addcontentsline{toc}{subsection}{\thesubsection. Installation and Quick Start}

The remaining assumes that \pkg{pbdMPI} is installed correctly.
If \pkg{pbdMPI} is not yet installed, see the \pkg{pbdMPI}
vignette~\citep{Chen2012pbdMPIvignette} for installation details.
Users can download \pkg{pbdSLAP} from CRAN at
\url{http://cran.r-project.org}, and
the installation can be done with the following commands
\begin{Command}
tar zxvf pbdMPI_0.1-0.tar.gz
R CMD INSTALL pbdSLAP
\end{Command}

Users can get started quickly with \pkg{pbdSLAP} by learning from the
following example.
\begin{Command}
### Under command mode, run the demo with 2 processors by
### (Use Rscript.exe for windows system)

mpiexec -np 2 Rscript -e "demo(gridinfo,'pbdSLAP',ask=F,echo=F)"
\end{Command}

