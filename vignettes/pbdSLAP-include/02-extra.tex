
\section[Useful Information]{Useful Information}
\label{sec:useful_info}
\addcontentsline{toc}{section}{\thesection. Useful Information}


\subsection[\code{CDEFS}~Flag for ScaLAPACK]{\code{CDEFS}~Flag of ScaLAPACK}
\label{sec:compilers}
\addcontentsline{toc}{subsection}{\thesubsection. \code{CDEFS}~Flag of ScaLAPACK}

The \code{CDEFS} is a flag for interface between C and Fortran, and is
required to compile \pkg{pbdSLAP}.
\code{CDEFS = -DAdd_} is the default for GNU C and Fortran.
The correct value can be determined by BLACS test intface
tool in \code{pbdSLAP/inst/intface/}.
For other compilers, the possible values could be \code{-DNoChange},
\code{-Df77IsF2C}, or \code{-DUpCase}.

For other values, user can provide an external \code{CDEFS} flag at
installation time, such as \slapversion
\begin{Command}[escapechar=\^]
tar zxvf pbdSLAP_^\slapversion^.tar.gz
R CMD INSTALL pbdSLAP \
  --configure-vars="CDEFS='other possible flags'"
\end{Command}
The value of \code{CDEFS} (other possible flags) can be
one of the possible values above or other configurations.
For PGI compiler, the value could be \code{-DAdd_ -DNO_IEEE $(USEMPI)}.%$
Note that the \code{CDEFS} will overwrite the default \code{CDEFS}
inside \code{pbdSLAP/src/Makevars.in} for compiling \code{libslap.a}.


\subsection[Using Other Distributions of ScaLAPACK]{Using Other Distributions of ScaLAPACK}
\label{sec:external}
\addcontentsline{toc}{subsection}{\thesubsection. Using Other Distributions of ScaLAPACK}

Some users may have access to other, often non-free, distributions of
Scalapack, such as Intel's \pkg{MKL}
(\url{http://software.intel.com/en-us/intel-mkl}) or Cray's \pkg{LibSci}
(\url{http://docs.cray.com/}).  It may be possible to achieve some
performance gains using these libraries over netlib ScaLAPACK, especially
when using big machines, such as Nautilus and Kraken in
NICS (\url{http://www.nics.tennessee.edu}).

To use these external libraries with \pkg{pbdSLAP}, you need only supply the
appropriate flag to \code{EXT_LDFLAGS} at compile time.  Specifically,
the user would issue the command:
\begin{Command}[escapechar=\^]
tar zxvf pbdSLAP_^\slapversion^.tar.gz
R CMD INSTALL pbdSLAP \
  --configure-vars="EXT_LDFLAGS='external ldflags'"
\end{Command}
where "external ldflags" needs to include linking information to
ScaLAPACK, BLACS, LAPACK, and BLAS.
Please be aware that the order matters in the \code{EXT_LDFLAGS}.
Note that \code{EXT_LDFLAGS} will be part of \code{PKG_LDFLAGS}
inside \code{pbdSLAP/src/Makevars.in}.
